\documentclass[A4,11pt]{article}
%\documentclass[letterpaper,11pt]{article} %For use in US
\usepackage{latexsym}
\usepackage[empty]{fullpage}
\usepackage{titlesec}
\usepackage{marvosym}
\usepackage[usenames,dvipsnames]{color}
\usepackage{verbatim}
\usepackage{enumitem}
\usepackage[hidelinks]{hyperref}
\usepackage[english]{babel}
\usepackage{tabularx}
\usepackage{tikz}
\input{glyphtounicode}

\begin{comment}
I am by no means a professional when it comes to the CV's/resumes, I have
received various trainings on how to write a CV and resume from my high 
school, as well as the Austin College and University of Eastern Finland's
career counseling departments. As I intend to share my CV as a template, I 
feel that it is my responsibility to provide explanations of my work.
\end{comment}


%-----FONT OPTIONS-------------------------------------------------------------
\begin{comment}
The font of the document will impact not just how readable it is, but how it is
perceived. In the "The Craft of Scientific Writing" by Michael Alley, shares a
common fonts for publication as well as their use. I have chosen to use
Palatino for its legibility, some others are given below. There is far too much
about typography to discus here. Note: serif fonts have short projecting
strokes, sans-serif fonts are sans (without) these strokes.
\end{comment}


% serif
 \usepackage{palatino}
% \usepackage{times} %This is the default as well
% \usepackage{charter}

% sans-serif
% \usepackage{helvet}
% \usepackage[sfdefault]{noto-sans}
% \usepackage[default]{sourcesanspro}

%-----PAGE SETUP---------------------------------------------------------------

% Adjust margins
\addtolength{\oddsidemargin}{-1cm}
\addtolength{\evensidemargin}{-1cm}
\addtolength{\textwidth}{2cm}
\addtolength{\topmargin}{-1cm}
\addtolength{\textheight}{2cm}

% Margins for US Letter size
%\addtolength{\oddsidemargin}{-0.5in}
%\addtolength{\evensidemargin}{-0.5in}
%\addtolength{\textwidth}{1in}
%\addtolength{\topmargin}{-.5in}
%\addtolength{\textheight}{1.0in}

\urlstyle{same}

\raggedbottom
\raggedright
\setlength{\tabcolsep}{0cm}

% Sections formatting
\titleformat{\section}{
  \vspace{-4pt}\scshape\raggedright\large
}{}{0em}{}[\color{black}\titlerule \vspace{-5pt}]

% Ensure that .pdf is machine readable/ATS parsable
\pdfgentounicode=1

%-----CUSTOM COMMANDS FOR FORMATTING SECTIONS----------------------------------
\newcommand{\CVItem}[1]{
  \item\small{
    {#1 \vspace{-2pt}}
  }
}

\newcommand{\CVSubheading}[4]{
  \vspace{-2pt}\item
    \begin{tabular*}{0.97\textwidth}[t]{l@{\extracolsep{\fill}}r}
      \textbf{#1} & #2 \\
      \small#3 & \small #4 \\
    \end{tabular*}\vspace{-7pt}
}

\newcommand{\CVSubSubheading}[2]{
    \item
    \begin{tabular*}{0.97\textwidth}{l@{\extracolsep{\fill}}r}
      \text{\small#1} & \text{\small #2} \\
    \end{tabular*}\vspace{-7pt}
}

\newcommand{\CVSubItem}[1]{\CVItem{#1}\vspace{-4pt}}

\renewcommand\labelitemii{$\vcenter{\hbox{\tiny$\bullet$}}$}

\newcommand{\CVSubHeadingListStart}{\begin{itemize}[leftmargin=0.5cm, label={}]}
% \newcommand{\resumeSubHeadingListStart}{\begin{itemize}[leftmargin=0.15in, label={}]} % Uncomment for US
\newcommand{\CVSubHeadingListEnd}{\end{itemize}}
\newcommand{\CVItemListStart}{\begin{itemize}}
\newcommand{\CVItemListEnd}{\end{itemize}\vspace{-5pt}}

%------------------------------------------------------------------------------
% CV STARTS HERE  %
%------------------------------------------------------------------------------
\begin{document}

%-----HEADING------------------------------------------------------------------
\begin{comment}
In Europe it is common to include a picture of ones self in the CV. Select
which heading appropriate fr the document you are creating.
\end{comment}

\begin{minipage}[c]{0.05\textwidth}
\-\
\end{minipage}
\begin{minipage}[c]{0.2\textwidth}
\begin{tikzpicture}
    \clip (0,0) circle (1.75cm);
    \node at (0,-.5) {\includegraphics[width = 4cm]{portrait.jpeg}}; 
    % if necessary the picture may be moved by changing the at (coordinates)
    % width defines the 'zoom' of the picture
\end{tikzpicture}
\hfill\vline\hfill
\end{minipage}
\begin{minipage}[c]{0.4\textwidth}
    \textbf{\Huge \scshape{Sasa Salmen}} \\ \vspace{1pt}
    % \scshape sets small capital letters, remove if desired
    \small{+55 94 99136-2886} \\
    \href{mailto:you@provider.com}{\underline{sasinhe@gmail.com}}\\
    % Be sure to use a professional *personal* email address
    \href{https://github.com/sasinhe}{\underline{github.com/sasinhe}}
\end{minipage}

% Without picture
%\begin{center}
%    \textbf{\Huge \scshape Charles Rambo} \\ \vspace{1pt} %\scshape sets small capital letters, remove if desired
%    \small +1 123-456-7890 $|$ 
%    \href{mailto:you@provider.com}{\underline{you@provider.com}} $|$\\
%    % Be sure to use a professional *personal* email address
%    \href{https://linkedin.com/in/your-name-here}{\underline{linkedin.com/in/charles-rambo}} $|$
%    % you should adjust you linked in profile name to be professional and recognizable
%    \href{https://github.com/fizixmastr}{\underline{github.com/fizixmastr}}
%\end{center}



\begin{comment}
This CV was written for specifically for positions I was applying for in
academia, and then modified to be a template.

A standard CV is about two pages long where as a resume in the US is one page.
sections can be added and removed here with this in mind. In my experience, 
education, and applicable work experience and skills are the most import things
to include on a resume. For a CV the Europass CV suggests the categories: Work
Experience, Education and Training, Language Skills, Digital Skills,
Communication and Interpersonal Skills, Conferences and Seminars, Creative Works
Driver's License, Hobbies and Interests, Honors and Awards, Management and
Leadership Skills, Networks and Memberships, Organizational Skills, Projects,
Publications, Recommendations, Social and Political Activities, Volunteering.

Your goal is to convey a who, what , when, where, why for every item you share. 
The who is obviously you, but I believe the rest should be done in that order.
For example below. An employer cares most about the degree held and typically 
less about the institution or where it is located (This is still good 
information though). Whatever order you choose be consistent throughout.
\end{comment}

%-----EDUCATION----------------------------------------------------------------
\section{Education}
  \CVSubHeadingListStart
%    \CVSubheading % Example
%      {Degree Achieved}{Years of Study}
%      {Institution of Study}{Where it is located}
    
    \CVSubheading
    {B.S. Molecular Sciences}{Jul. 2019 - current}
    {University of São Paulo}{São Paulo, SP}
    
    \CVSubheading
    {B.S. Physics [interrupted]}{Jan. 2019 -- Jul. 2019}
      {University of São Paulo}{São Paulo, SP}
  \CVSubHeadingListEnd

%-----WORK EXPERIENCE----------------------------------------------------------
\begin{comment}
try to briefly explain what you did and why it is relevant to the position you
are seeking
\end{comment}

\section{Work Experience}
  \CVSubHeadingListStart
%    \CVSubheading %Example
%      {What you did}{When you worked there}
%      {Who you worked for}{Where they are located}
%      \CVItemListStart
%        \CVItem{Why it is important to this employer}
%      \CVItemListEnd
    \CVSubheading
      {Freelancing Teacher}{Jan. 2019 -- Jul. 2020}
      {Centro de Formação Laplace}{São Paulo, SP}
      \CVItemListStart
      \CVItem{Worked developing solutions of previous Academic Olympics (like OBMEP, OBM, OBF\ldots) in LaTeX}
      \CVItemListEnd
    \CVSubheading
      {I.T Assistant}{Jul. 2018 -- Dec. 2018}
      {IABC}{São Paulo, SP}
      \CVItemListStart
        \CVItem{Did mateinance on several desktop computers of the school}
        \CVItem{Helped developing solutions to the schools network firewall}
        \CVItem{Did some cameraman jobs on real time transmissions of the church}
    \CVItemListEnd
  \CVSubHeadingListEnd

%-----PROJECTS AND RESEARCH----------------------------------------------------
\begin{comment}
Ideally the title of the work should speak for what it is. However if you feel
like you should explain more about why the project is applicable to this job,
use item list as is shown in the work experience section.
\end{comment}

\section{Projects}
  \CVSubHeadingListStart
%    \CVSubheading
%      {Title of Work}{When it was done}
%      {Institution you worked with}{unused}
     \CVSubheading
     {Project Euler | \textit{Python and Haskell}}{2021}
     {By myself}{}

  \CVSubHeadingListEnd

  \section{Research Topics}
  \CVSubHeadingListStart
    \CVSubheading
    {Functional Analysis}{2019}
    {With Professor Ricardo Pereira (IME-USP)}{}

    \CVSubheading
    {Lattice QFT}{2022-}
    {With Professor Tereza Mendes (IFSC-USP)}{}

  \CVSubHeadingListEnd

%-----SKILLS-------------------------------------------------------------------
\begin{comment}
This section is compressed from the various skills sections that Euro CV
recommends.
\end{comment}

\section{Skills}
 \begin{itemize}[leftmargin=0.5cm, label={}]
    \small{\item{
            \textbf{Languages}{: Portugese (Native), English (Fluent), Spanish (Basic)} \\
            \textbf{Programming}{: GNU/Linux, Python (matplotlib, numpy, scipy, pandas), Mathematica, Vimscript, Elisp, Java, C/C++, git} \\
         \textbf{Document Creation}{: Microsoft Office Suite, LaTeX, HTML/CSS} \\
         \textbf{Knowledge Areas}{: Quantum Mechanics, Quantum Info, Calculus (I-IV), Probability and Statistics, Numerical Analysis, QFT (in progress)} \\
         }}


 \end{itemize}
    
%------------------------------------------------------------------------------
 \end{document}
